\documentclass{ctexart}
\usepackage[a4paper]{geometry}

\begin{document}

\title{项目申请书}
\author{向嘉豪}
\date{\today}

\maketitle


\section{立项依据}
\subsection{项目的研究目的、意义}

\subsubsection{项目的研究目的}
随着基础通信设施的不断完善,如5G网络的普及,人们对通信的需求逐渐增加。这种需求的增长催生了大量基于通信的应用,如工业互联网、车联网、物联网等。
然而,随着这些网络中传输的数据量的增加,安全问题变得越来越严重。这主要是由网络数据中数据包传输机制引起的。幸运的是,对称加密算法提供了一种解决方案,只需双方协商好密钥,就能确保数据在公共信道下的安全传输,而无需调整传输机制。 

对称加密算法中,最广泛使用的是2001年的AES[1]国际算法。它在WEB、WIFI等领域,以及服务器与个人计算机上都得到了广泛的应用。然而,为了避免AES算法存在的未知门陷,我国在2006年提出了SMS4[2]对称加密算法来替代AES算法。值得一提的是,SMS4在2021年正式成为ISO标准并得到了国际认可。
这些传统的对称算法在计算资源充足的场景下,能够提供较高的安全性。但在计算资源受限的场景下,由于这些算法的计算复杂度较高,因此需要一种更加轻量级的对称加密算法来满足这种场景的需求。

美国国家标准局NIST于2019年启动了轻量级密码算法LWC的征召。在56个轻量级密码算法的竞争中,ASCON[3]算法经过三轮筛选,最终脱颖而出。到了2023年,ASCON算法已经成为NIST的轻量级加密标准。由于ASCON算法在资源受限的场景下表现出较高的性能,因此在物联网、车联网等场景中具有广泛的应用前景。

相较而言,我国在轻量级加密算法上的起步较晚,暂时还没有自己的轻量级加密算法标准。然而,在2019年由中国密码学会组织的全国密码算法设计竞赛中,一等奖获得者是一种名为uBlock[4]的轻量级加密算法。这种算法在计算资源充足的环境下表现优秀,同时在资源受限的场景下,也展现出了高性能。尽管如此,uBlock算法并未像ASCON算法那样得到广泛的学者关注,这使得我国在轻量级加密算法上与国际顶尖水平存在较大的差距。

这种差距不仅体现在轻量级加密算法的设计上,也同样存在于其实现上。这包括硬件和软件实现。具体来说,需要考虑如何高效设计硬件加密的IP核,并将其集成入芯片,以实现硬件级别的安全。同时,也需要考虑如何将加密算法高效实现于8-bit或32-bit的微控制器中,以确保应用的软件级别安全。我国在这个领域的研究相对较少。因此,本项目的目标是研究轻量级加密算法的实现,以推动我国在轻量级加密算法领域的研究进展。

\subsubsection{项目的研究意义}
在上世纪的算法设计中,对称加密算法的设计主要考虑了安全性。然而,这些设计往往较少考虑其在资源受限的场景下的性能。例如,DES[6]的实现部分提及其在软件与硬件的实现,但并未给出具体的参考实现方案与实现性能。
在NIST的LWC竞赛中,轻量级加密算法的设计不仅考虑了安全性,还考虑了其在资源受限的场景下的实现性能。这无疑对算法的设计提出了更高的要求。实现算法设计与实现性能之间的桥梁,是本项目研究的出发点。
在考虑最新的软硬件平台技术下,将加密算法的组件转化为相应的电路或程序,是本项目研究的手段。更具体来说,本项目的研究意义体现在以下几个方面:

1. 研究轻量级加密算法在专用集成电路(ASCI)和现场可编程门阵列(FPGA)上的硬件实现。这将提高算法实现的性能,同时确保其硬件级别的安全。
2. 研究轻量级加密算法在8-bit或32-bit的微控制器上的软件实现。这将提高算法实现的性能,同时确保其软件级别的安全。
3. 研究轻量级加密算法的软硬件协同实现。在确保算法实现的灵活性的前提下,最大程度地提高算法实现的性能。


\subsection{国内外研究现状分析和发展趋势}

网络数据传输量的增加使得网络安全问题日益严重。对称加密算法是一种能够确保数据在公共信道下安全传输的解决方案。然而,传统的对称加密算法在计算资源受限的场景下计算复杂度较高。因此,需要一种轻量级的对称加密算法来满足这种场景的需求。

在2011年之前,轻量级加密算法的研究主要集中在硬件实现上。这是因为硬件实现能够提供更高的性能,如ISO的轻量级加密标准PRESENT[7]。同时,硬件实现也能确保算法的安全性。
然而,随着应用场景的增加,性能需求也在变化。现在,不仅关注硬件的电路面积,也开始关注软件加密的执行时间,如美国国家安全局(NSA)2013年提出的SIMON[8]。此外,还关注硬件加密的功耗,如2015年亚密会提出的Midori[9],以及硬件加密的时延,如2016年美密会提出的SKINNY[10]。因此,轻量级加密算法的内涵正在不断扩展。

在轻量级加密算法设计不断


\subsection{参考文献}
[1] Daemen, Joan, and Vincent Rijmen. "AES proposal: Rijndael." (1999).

[2] Diffie, Whitfield, and George Ledin. "SMS4 encryption algorithm for wireless networks." Cryptology ePrint Archive (2008).

[3] Dobraunig, Christoph, et al. "Ascon v1. 2: Lightweight authenticated encryption and hashing." Journal of Cryptology 34 (2021): 1-42.

[4] Wen-Ling, Wu, et al. "The block cipher uBlock." Journal of Cryptologic Research 6.06 (2019): 690-703.

[5] Pub, F. I. P. S. "Data encryption standard (des)." FIPS PUB (1999): 46-3.

[6] Mohajerani, Kamyar, et al. "FPGA benchmarking of round 2 candidates in the NIST lightweight cryptography standardization process: Methodology, metrics, tools, and results." Cryptology ePrint Archive (2020).

[7] Bogdanov, Andrey, et al. "PRESENT: An ultra-lightweight block cipher." Cryptographic Hardware and Embedded Systems-CHES 2007: 9th International Workshop, Vienna, Austria, September 10-13, 2007. Proceedings 9. Springer Berlin Heidelberg, 2007.

[8] Beaulieu, Ray, et al. "The SIMON and SPECK lightweight block ciphers." Proceedings of the 52nd annual design automation conference. 2015.

[9] Banik, Subhadeep, et al. "Midori: A block cipher for low energy." Advances in Cryptology–ASIACRYPT 2015: 21st International Conference on the Theory and Application of Cryptology and Information Security, Auckland, New Zealand, November 29--December 3, 2015, Proceedings, Part II 21. Springer Berlin Heidelberg, 2015.

[10] Beierle, Christof, et al. "The SKINNY family of block ciphers and its low-latency variant MANTIS." Advances in Cryptology–CRYPTO 2016: 36th Annual International Cryptology Conference, Santa Barbara, CA, USA, August 14-18, 2016, Proceedings, Part II 36. Springer Berlin Heidelberg, 2016.

\subsection{项目应用前景和学术价值}
\subsection{现有研究基础、条件、手段}
\subsubsection{现有研究基础}
\subsubsection{现有研究条件}
\subsubsection{现有研究手段}

\section{研究方案}
\subsection{研究目标、研究内容和拟解决的关键问题}
\subsubsection{研究目标}
\subsubsection{研究内容}
\subsubsection{拟解决的关键问题}
\subsection{拟采取的研究方法及可行性分析}
\subsubsection{拟采取的研究方法}
\subsubsection{可行性分析}

\subsection{本项目的创新之处}
\subsection{预期研究进展}


\end{document}