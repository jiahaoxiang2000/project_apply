\documentclass{ctexart}
\usepackage[a4paper]{geometry}

\begin{document}

\title{项目申请书}
\author{向嘉豪}
\date{\today}

\maketitle


\section{立项依据}
\subsection{项目的研究目的、意义}

\subsubsection{项目的研究目的}
随着基础通信设施的不断完善,如5G网络的普及,人们对通信的需求逐渐增加。这种需求的增长催生了大量基于通信的应用,如工业互联网、车联网、物联网等。
然而,随着这些网络中传输的数据量的增加,安全问题变得越来越严重。这主要是由网络数据中数据包传输机制引起的。幸运的是,对称加密算法提供了一种解决方案,只需双方协商好密钥,就能确保数据在公共信道下的安全传输,而无需调整传输机制。 

对称加密算法中,最广泛使用的是2001年的AES[1]国际算法。它在WEB、WIFI等领域,以及服务器与个人计算机上都得到了广泛的应用。然而,为了避免AES算法存在的未知门陷,我国在2006年提出了SM4[2]对称加密算法来替代AES算法。值得一提的是,SM4在2021年正式成为ISO标准并得到了国际认可。
这些传统的对称算法在计算资源充足的场景下,能够提供较高的安全性。但在计算资源受限的场景下,由于这些算法的计算复杂度较高,因此需要一种更加轻量级的对称加密算法来满足这种场景的需求。

美国国家标准局NIST于2019年启动了轻量级密码算法LWC的征召。在56个轻量级密码算法的竞争中,ASCON[3]算法经过三轮筛选,最终脱颖而出。到了2023年,ASCON算法已经成为NIST的轻量级加密标准。由于ASCON算法在资源受限的场景下表现出较高的性能,因此在物联网、车联网等场景中具有广泛的应用前景。

相较而言,我国在轻量级加密算法上的起步较晚,暂时还没有自己的轻量级加密算法标准。然而,在2019年由中国密码学会组织的全国密码算法设计竞赛中,一等奖获得者是一种名为uBlock[4]的轻量级加密算法。这种算法在计算资源充足的环境下表现优秀,同时在资源受限的场景下,也展现出了高性能。尽管如此,uBlock算法并未像ASCON算法那样得到广泛的学者关注,这使得我国在轻量级加密算法上与国际顶尖水平存在较大的差距。

这种差距不仅体现在轻量级加密算法的设计上,也同样存在于其实现上。这包括硬件和软件实现。具体来说,需要考虑如何高效设计硬件加密的IP核,并将其集成入芯片,以实现硬件级别的安全。同时,也需要考虑如何将加密算法高效实现于8-bit或32-bit的微控制器中,以确保应用的软件级别安全。我国在这个领域的研究相对较少。因此,本项目的目标是研究轻量级加密算法的实现,以推动我国在轻量级加密算法领域的研究进展。

\subsubsection{项目的研究意义}
在上世纪的算法设计中,对称加密算法的设计主要考虑了安全性。然而,这些设计往往较少考虑其在资源受限的场景下的性能。例如,DES[6]的实现部分提及其在软件与硬件的实现,但并未给出具体的参考实现方案与实现性能。
在NIST的LWC竞赛中,轻量级加密算法的设计不仅考虑了安全性,还考虑了其在资源受限的场景下的实现性能。这无疑对算法的设计提出了更高的要求。实现算法设计与实现性能之间的桥梁,是本项目研究的出发点。
在考虑最新的软硬件平台技术下,将加密算法的组件转化为相应的电路或程序,是本项目研究的手段。更具体来说,本项目的研究意义体现在以下几个方面:

1. 研究轻量级加密算法在专用集成电路(ASCI)和现场可编程门阵列(FPGA)上的硬件实现。这将提高算法实现的性能,同时确保其硬件级别的安全。
2. 研究轻量级加密算法在8-bit或32-bit的微控制器上的软件实现。这将提高算法实现的性能,同时确保其软件级别的安全。
3. 研究轻量级加密算法的软硬件协同实现。在确保算法实现的灵活性的前提下,最大程度地提高算法实现的性能。


\subsection{国内外研究现状分析和发展趋势}

\subsection{参考文献}
[1] Daemen, Joan, and Vincent Rijmen. "AES proposal: Rijndael." (1999).

[2] Diffie, Whitfield, and George Ledin. "SMS4 encryption algorithm for wireless networks." Cryptology ePrint Archive (2008).

[3] Dobraunig, Christoph, et al. "Ascon v1. 2: Lightweight authenticated encryption and hashing." Journal of Cryptology 34 (2021): 1-42.

[4] Wen-Ling, Wu, et al. "The block cipher uBlock." Journal of Cryptologic Research 6.06 (2019): 690-703.

[5] Pub, F. I. P. S. "Data encryption standard (des)." FIPS PUB (1999): 46-3.

[6] Mohajerani, Kamyar, et al. "FPGA benchmarking of round 2 candidates in the NIST lightweight cryptography standardization process: Methodology, metrics, tools, and results." Cryptology ePrint Archive (2020).


\subsection{项目应用前景和学术价值}
\subsection{现有研究基础、条件、手段}
\subsubsection{现有研究基础}
\subsubsection{现有研究条件}
\subsubsection{现有研究手段}

\section{研究方案}
\subsection{研究目标、研究内容和拟解决的关键问题}
\subsubsection{研究目标}
\subsubsection{研究内容}
\subsubsection{拟解决的关键问题}
\subsection{拟采取的研究方法及可行性分析}
\subsubsection{拟采取的研究方法}
\subsubsection{可行性分析}

\subsection{本项目的创新之处}
\subsection{预期研究进展}


\end{document}